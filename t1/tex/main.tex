\documentclass{article}

\usepackage[utf8]{inputenc}

\usepackage{scrextend}

\usepackage{geometry}
\geometry{
	a4paper,
	total={170mm,257mm},
	left=20mm,
	top=20mm,
}

\usepackage{graphicx}
\usepackage[portuguese]{babel}
\usepackage{subfig}

	
\usepackage{listings}
\usepackage{xcolor}

\definecolor{codegreen}{rgb}{0,0.6,0}
\definecolor{codegray}{rgb}{0.5,0.5,0.5}
\definecolor{codepurple}{rgb}{0.58,0,0.82}
\definecolor{backcolour}{rgb}{0.95,0.95,0.92}

\lstdefinestyle{mystyle}{
	backgroundcolor=\color{backcolour},   
	commentstyle=\color{codegreen},
	keywordstyle=\color{magenta},
	numberstyle=\tiny\color{codegray},
	stringstyle=\color{codepurple},
	basicstyle=\ttfamily\footnotesize,
	breakatwhitespace=false,         
	breaklines=true,                 
	captionpos=b,                    
	keepspaces=true,                 
	numbers=left,                    
	numbersep=5pt,                  
	showspaces=false,                
	showstringspaces=false,
	showtabs=false,                  
	tabsize=2
}

\lstset{style=mystyle}

%\usepackage{indentfirst}
\setlength{\parindent}{1.5cm}% too much in my eyes delete this
% line and use the default ...

\usepackage{indentfirst}


\title{
	Ética \& Inteligência Artificial (Trabalho 1) \\
	\Large Aprendizado de Máquinas e Reconhecimento de Padrões (MO886) \\
	Randerson A. Lemos (103897)
}

\date{\vspace{-5ex}}

\begin{document}
  \pagenumbering{gobble}
  \maketitle

%
%%
\section*{Introdução}

Os avanços das tecnologias computacionais dos últimos anos somados ao desenvolvimento e ao aprimoramento de novos algoritmos de inteligência artificial possibilitaram a resolução de problemas e automatização de processos e tomadas de decisões em diversos campos da vida cotidiana tanto dentro dos espaços privadas quanto públicos. A consolidação dos algoritmos de aprendizado de máquinas - com destaque para as redes neurais - e a maior disponibilidade de dados nos seus mais diversos formatos (categórico, numérico, fotográfico, auditivo) impulsionaram diversas áreas sensíveis a sociedade moderna, como por exemplo: do transporte com os veículos autônomos, da agricultura com identificação de pragas, da saúde com o diagnóstico de doenças, da segurança com sistemas anti-fraude e de reconhecimento facial, do entretenimento e lazer com os algoritmos de sugestão de conteúdos e produtos personalizados, entre outros \cite{Ludermir_2021}.

Diante de tantas possibilidades de aplicação e uso, é possível afirmar que os algoritmos de inteligência firmaram um posição bastante próxima junto a sociedade modernada de maneira geral e na vida das pessoas de maneira particular. A interação próxima entre homem e máquina ou melhor entre homem e algoritmos de inteligência artificial, cuja tendência e se tornar cada vez mais íntima com o passar do tempo, suscita diversas discussões e questionamentos sobre os efeitos colaterais advindos da relação estabelecida entre esses dois atores. Uma dessas discussões, é centrada na ética na inteligência artificial e busca entender como esse conceito, que foi primordialmente concebido para operar entre as relações de homem para homem pode ser aplicado nas relações de homem para máquina \cite{Ludermir_2021}. 

Com o objetivo de ser uma introdução a discussão da ética na inteligência artificial, esse trabalho traz a seguir os seguintes tópicos:
\begin{itemize}
	\item uma definição de ética na inteligência artificial;
	\item apresentação de um notícia recente (2022) de um problema de ética na inteligência artificial; e
	\item uma breve descrição dos problemas causados pelas `Armas de Destruição em Massa' do capítulo 4 do livro `Algoritmos de Destruição em Massa'.
\end{itemize}


\section*{Definição de Ética na Inteligência Artificial}
De acordo com a versão online do dicionário Aurélio \cite{aurelio_online}, o termo ética é primeiramente definido como uma área de estudo da filosofia que se dedica a entender e analisar as motivações que ocasionam, alteram ou orientam o comportamento humano, principalmente as que estão conformadas por regras, preceitos ou normas sociais. Dessa maneira, ainda de acordo com o dicionário Aurélio \cite{aurelio_online}, a ética pode ser entendida como a reunião de normas de juízo de valor moral presentes nas pessoas, em uma sociedade ou grupo social. 

Com a consolidação das soluções de inteligência artificial em diversos âmbitos da vida humana, o assunto sobre ética na inteligência artificial vem atraído cada vez mais interesse da sociedade e hoje é uma parte a mais da área de estudo da ética. Um dos focos da ética na inteligência artificial é o de entender a ética das máquinas e suas condutas para que a interação delas com os seres humanos, ou seres vivos de maneira mais ampla, não seja danosa, 

De maneira simplista, a inteligência artificial se dedica a desenvolver máquinas e algoritmos que sejam capazes de tomarem decisões coerentes e satisfatórias de acordo com informações de entrada e um contexto. Decisões coerentes e satisfatórias podem ser entendidas como aquelas que seriam igualmente tomadas por seres humanos capacitados para tal. Aqui, novamente, o assunto da ética salta aos olhos e com ele as chamadas Leis da Robótica. As três leis da robótica foram idealizadas pelo escrito Isaac Asimov e tem como objetivo balizar certos comportamentos das máquinas de modo a proteger a próprio ser humano. Essas regas são:
\begin{itemize}
	\item Primeira lei: um robô não pode ferir um ser humano ou, por inação, permitir que um ser humano sofre algum mal;
	\item segunda lei: um robô deve obedecer as ordens que lhe sejam dadas por seres humanos exceto nos casos em que tais ordens entrem em conflito com a Primeira lei; e
	\item Terceira lei: um robô deve proteger sua própria existência desde que tal proteção não entre em conflito com a Primeira ou a Segunda leis.
\end{itemize}

\section*{Notícia envolvendo Ética e Inteligência Artificial}
Em uma busca rápida pela internet sobre o assunto de ética e inteligência artificial diversas notícias, somente deste ano de 2022, foram encontradas. Se, por um lado, a quantidade de notícias encontradas não foi algo surpreendente, dado que esse assunto é realmente atual, por outro lado, o conteúdo das notícias foi, sem dúvidas, surpreendente. Isso porque muitas dessas notícias abordavam experiências não satisfatórias no campo da ética entre homens e máquinas. Uma das notícias encontradas foi veiculada pelo Jornal Estado de São Paulo (O ESTADÃO) e apresentava o seguinte título: Uma inteligência artificial sem ética pode arruinar a sociedade \cite{Estadao_online}. No início, há na notícia a afirmação de que os sistemas de inteligência artificial aprendem conosco e, na sequência, a pergunta ``Mas será que estamos sendo bons professores?". A explicação da pergunta, aparentemente sem muito nexo, vem em seguida: a ideia é contextualizar uma discussão sobre ética na inteligência artificial voltada para os riscos que estão suscetíveis os sistemas inteligentes de aprenderem comportamentos preconceituosos e não mais toleráveis dentro da nossa sociedade. Risco que é agravado quando levado em conta que muitos dessas sistemas inteligentes interagirão com multidões e, assim, poderão passar adiante os comportamentos nocivos para outras pessoas. Como forma de ilustrar as razões da discussão levantada, na notícia há apresentação do caso da ferramenta Tay, que foi lançada pela Microsoft. Essa ferramenta ficou por traz de uma conta no Twitter e buscava simular uma adolescente criada para conversar com os usuários. A ferramente não ficou no ar por mais de 24 horas. Isso porque nesse pequeno tempo, depois de interagir e conversar com diversas pessoas, a ferramenta desenvolveu uma personalidade racista, xenofóbica e sexista. A Tay, por exemplo, se tornou fã de Adolf Hitler e seus ideias. Na mesma notícia um outro problema associado as soluções de inteligência artificial e ética que acontece nos sistemas de recrutamento foi apresentado. Esses sistemas analisam milhares de currículos para uma vaga e, se não estiverem bem treinados, podem deixar de fora bons profissionais devido a algum viés desenvolvido com relação a idade, gênero ou raça dos candidatos.



\section*{Problemas das Armas de Destruição em Massa}


\bibliographystyle{plain} % We choose the "plain" reference style
\bibliography{refs} % Entries are in the refs.bib file

\end{document}