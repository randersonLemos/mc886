\documentclass{article}

\usepackage[utf8]{inputenc}

\usepackage{scrextend}

\usepackage{geometry}
\geometry{
	a4paper,
	total={170mm,257mm},
	left=20mm,
	top=20mm,
}

\usepackage{graphicx}
\usepackage[portuguese]{babel}
\usepackage{subfig}

	

\usepackage{listings}
\usepackage{xcolor}

\definecolor{codegreen}{rgb}{0,0.6,0}
\definecolor{codegray}{rgb}{0.5,0.5,0.5}
\definecolor{codepurple}{rgb}{0.58,0,0.82}
\definecolor{backcolour}{rgb}{0.95,0.95,0.92}

\lstdefinestyle{mystyle}{
	backgroundcolor=\color{backcolour},   
	commentstyle=\color{codegreen},
	keywordstyle=\color{magenta},
	numberstyle=\tiny\color{codegray},
	stringstyle=\color{codepurple},
	basicstyle=\ttfamily\footnotesize,
	breakatwhitespace=false,         
	breaklines=true,                 
	captionpos=b,                    
	keepspaces=true,                 
	numbers=left,                    
	numbersep=5pt,                  
	showspaces=false,                
	showstringspaces=false,
	showtabs=false,                  
	tabsize=2
}

\lstset{style=mystyle}

%\usepackage{indentfirst}
\setlength{\parindent}{1.5cm}% too much in my eyes delete this
% line and use the default ...

\usepackage{indentfirst}


\title{
	Ética \& Inteligência Artificial (Trabalho 1) \\
	\Large Aprendizado de Máquinas e Reconhecimento de Padrões (MO886) \\
	Randerson A. Lemos (103897)
}

\date{\vspace{-5ex}}

\begin{document}
  \pagenumbering{gobble}
  \maketitle

%
%%
\section*{Introdução}

Os avanços das tecnologias computacionais dos últimos anos somados ao desenvolvimento e ao aprimoramento de novos algoritmos de inteligência artificial possibilitaram a resolução de problemas e automatização de processos e tomadas de decisões em diversos campos da vida cotidiana tanto dentro dos espaços privadas quanto públicos. A consolidação dos algoritmos de aprendizado de máquinas - com destaque para as redes neurais - e a maior disponibilidade de dados nos seus mais diversos formatos (categórico, numérico, fotográfico, auditivo) impulsionaram diversas áreas sensíveis a sociedade moderna, como por exemplo: do transporte com os veículos autônomos, da agricultura com identificação de pragas, da saúde com o diagnóstico de doenças, da segurança com sistemas anti-fraude e de reconhecimento facial, do entretenimento e lazer com os algoritmos de sugestão de conteúdos e produtos personalizados, entre outros \cite{Ludermir_2021}.

Diante de tantas possibilidades de aplicação e uso, é possível afirmar que os algoritmos de inteligência firmaram um posição bastante próxima junto a sociedade modernada de maneira geral e na vida das pessoas de maneira particular. A interação próxima entre homem e máquina ou melhor entre homem e algoritmos de inteligência artificial, cuja tendência e se tornar cada vez mais íntima com o passar do tempo, suscita diversas discussões e questionamentos sobre os efeitos colaterais advindos da relação estabelecida entre esses dois atores. Uma dessas discussões, é centrada na ética na inteligência artificial e busca entender como esse conceito, que foi primordialmente concebido para operar entre as relações de homem para homem pode ser aplicado nas relações de homem para máquina \cite{Ludermir_2021}. 

Com o objetivo de ser uma introdução a discussão da ética na inteligência artificial, esse trabalho traz a seguir os seguintes tópicos:
\begin{itemize}
	\item uma definição de ética na inteligência artificial;
	\item apresentação de um noticia recente (2022) de um problema de ética na inteligência artificial; e
	\item uma breve descrição dos problemas causados pelas `Armas de Destruição em Massa' do capítulo 4 do livro `Algoritmos de Destruição em Massa'.
\end{itemize}


\section*{Definição de Ética na Inteligência Artificial}
De acordo com a versão online do dicionário Aurélio \cite{aurelio_online}, 

\section*{Notícia envolvendo Ética e Inteligência Artificial}


\section*{Problemas das Armas de Destruição em Massa}


\bibliographystyle{plain} % We choose the "plain" reference style
\bibliography{refs} % Entries are in the refs.bib file

\end{document}